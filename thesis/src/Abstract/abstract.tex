\begin{abstract}

  Brain–computer interfaces are already improving the circumstances of a small subset of the human population with neurological conditions (e.g. prosthetic control for paralysed patients). On the other hand, brain–computer interfaces for the general population could increase the potential for a better understanding of the brain due to generating more brain data. As a result, such mass-market brain–computer interfaces could improve the lives of healthy people through more natural or efficient interactions with technology or people around them or by directly altering human brains for certain benefits.

  In this bachelor’s thesis, the author describes the first steps toward a mass-market and generally applicable brain–computer interface software system that enables unidirectional neural communication with computers via the cloud. The author introduces the term neural/cloud interface in the current state of neuroscience research and the context of key industry players.

  Unlike in the natural sciences, a system such as a neural/cloud interface does not inherently occur, but is only achievable through engineering; thus, it must be created before it can be studied. Accordingly, through a case study project at the neurotechnology start-up IDUN Technologies, the reader is educated on the exact definition, motivation, project context and, most importantly, the results of building a neural/cloud interface system in practice. The central supposition is that a neural/cloud interface system may already be feasible with today’s software technologies and is not just speculation as in similar research on brain/cloud interfaces.

  In addition to the case study’s findings, an example of a cloud software architecture is developed and discussed in detail to securely stream, process and store brain data over the internet while respecting end-user privacy.

  Ultimately, this work aims to be a stepping stone into this new and interdisciplinary field between brain–computer interface software and cloud computing by providing a condensed overview of all critical aspects, insights and findings for future neural/cloud interface engineers. This would be achieved by building on the knowledge gathered during the implementation of this bachelor’s project.

\end{abstract}
