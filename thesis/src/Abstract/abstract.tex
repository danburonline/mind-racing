\begin{abstract}

Brain–computer interfaces are already improving the lives of a small subset of the human population with deficient living circumstances such as neurological conditions (e.g. paralysis). However, the prospect of a mass-market brain–computer interface for the general population holds many potential benefits for a better understanding of the brain due to a greater amount of neural data generated and, for example, an improvement in mental well-being and almost unlimited interaction possibilities through valuable insights from one of the last remaining black boxes of our body: the brain.

In this thesis, the author describes the first steps towards a mass-market and general-applicable brain–computer interface software system that enables communication with computers via the cloud for a production-ready end product. The term neural/cloud interface is introduced in the current state of neuroscience research and the context of key industry players.

Unlike in the natural sciences, a system such as a neural/cloud interface does not occur in nature but only in engineering; thus, it must be created before it can be studied. Accordingly, through a case study project at the neurotechnology start-up IDUN Technologies, the reader is educated on the exact definition, motivation, project context and, most importantly, the results of building a neural/cloud interface system in practice. The central assumption is that a neural/cloud interface system that enables the general applicability of brain–computer interface software for the mass market is already feasible with today's software technologies and is not just unrealistic speculation as in similar research on brain/cloud interfaces.

In addition to the case study findings, an exemplary cloud software architecture is developed and discussed in detail to securely stream, process and store neural data over the internet and cloud while respecting end-user privacy as part of the IDUN Technologies software offerings.

Ultimately, this work aims to be a stepping stone into this new and interdisciplinary field between brain–computer interface software and cloud computing by providing a condensed overview of all critical aspects, insights and findings for future neural/cloud interface engineers, building on the knowledge gathered during the implementation of this project.

\end{abstract}
