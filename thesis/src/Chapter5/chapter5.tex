\chapter{Results}
\graphicspath{{Chapter5/Figs/}{Chapter5/Figs/}}

% 5.1 Ergebnisse: Präsentation des konkreten Endergebnisses. Kompakte Zusammenfassung des Projekts unter Berücksichtigung der anfänglichen Zieldefinition. Wichtig ist dabei, dass man eine kritische Betrachtung der faktischen Resultate vornimmt (Evaluation). Hier ist ein Soll-Ist-Vergleich zur Zielsetzung aus Kapitel 1 mit kritischer Stellungnahme gewünscht. 5.2 Zusammenfassung: Es soll eine Zusammenfassung der Arbeit geschrieben werden und ein Fazit in Bezug auf das Projekt dessen Bedeutung (Relevanz und Nutzen) gezogen werden. Weiterhin soll eine Kritische Betrachtung der eigenen Vorgehensweise erfolgen. Abschließend soll ein Ausblick auf weitere Projektideen, die sich im Rahmen der Arbeit ergeben haben, gegeben werden (Folgeprojekte, Veröffentlichungen, Verwertung). Empfohlener Umfang: ca. 15-20%

\section{Steps Before The Analysis}
\label{steps-before-analysis}

% - Before we analysed the data, we removed all reaction times that were larger than 2000 ms
% (2\% of all observations) based on the assumption that such reaction times are unlikely to
% reflect spontaneous responses.
% - The data of two participants were excluded from the analyses because they did not complete
% the whole study.
% - Functional images were re-aligned, unwarped, corrected for slice timing, and spatially
% smoothed using an 8 mm smoothing kernel.

\section{Main Results}
\label{main-results}

% - First, we investigated whether X (research question)
% - We used an Independent samples t test with groups as independent variable and the
% depression score as dependent variable
% - The results showed that the difference between the groups/ conditions was significant
% - The results showed a significant correlation between…
% - The results showed a significant interaction between…
% - Specifically, the average depressions score was lower in the treatment group (M=3.45, SD = 2.18) compared to the placebo group (M=4.83, SD = 2.02).

\section{Figures And Tables}
\label{figures-and-tables}

% Add figures to make important results easier to interpret or to provide more information. Use tables to add extensive amounts of information that would be hard to read in text-form.

\section{Goals}
\label{chapter5-goals}

% o Did you describe everything that is needed to replicate your results?
% o Did you describe all pre-processing steps before the main analyses?
% o Did you mention to which research question each analysis belongs?
% o Did you avoid interpreting your results?
% o Did you add figures for making your key results easy to understand (or are they very simple)?
% o Did you add tables for extensive amounts of (numerical) information?

% o Does your discussion go from specific (interpretation) to broad (implications)?
% o Did you draw conclusions with reservations? (“A possible interpretation is…”)
% o If you expressed a preference for one explanation over another, did provide clear
% support for this preference?
% o Did you describe how your research connects to previous research?
% o Did you make clear what your research adds to existing research?
% o Did you describe how your research advance our understanding or how they may inspire
% future applications?
% o Did you clearly admit limitations before qualifying them?
% o Did you remind the reader of the value/implications of your research at the end?
% o Did you include some pointers for future research? (optional)

\section{Discussion}
\label{discussion}

\section{Conclusion}
\label{conclusion}

% - We investigated whether depression can be treated by training a positive focus
% - Our findings confirm this
% - Novel perspective on depression
% - More research needed, more treatments that follow this approach should be developed
