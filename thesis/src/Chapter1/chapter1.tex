\chapter{Introduction}
\graphicspath{{Chapter1/Figs/}{Chapter1/Figs/}}

\section{Background}
\label{chapter1-background}

There has been a long-standing interest in developing neural interfaces, systems that sense electrical impulses from the nervous system and use them to intercommunicate with the human brain. Successful research into the development of technologies that enable neural interfaces has been going on for decades \citep{vidal_real-time_1977}. Progress has accelerated significantly in the past few years, especially since the advent of modern processing capabilities such as in deep learning with convolutional neural networks (CNN) or generative adversarial networks (GAN) \citep{gonfalonieri_deep_2019}. In particular, a related discipline called brain-computer interfacing (BCI), a field focused on the direct interaction between brains and computers, has accumulated much momentum since the popularity of companies like Neuralink and Kernel.

One aspect of neural interfaces is hardware tailored to the human body. Whether it is an invasive sensor, such as in electrocorticography (ECoG), a method which uses electrodes placed on the surface of the brain, or a non-invasively placed sensor on the body, such as in electroencephalography (EEG). Both methods measure electrical activities produced by neurons; however, with decreasing spatial precision, the farther the electrode is placed from the brain, the more body structures (e.g. bones) are between firing neurons and the measuring sensor. The other aspect is software that reads and interprets data of these hardware sensors. Both aspects present their own set of challenges and complexities. Nonetheless, complete and applicable neural interfaces work in practice and have been used for many years in patients with neurological disorders \citep{braingate_publications_nodate}. There are also consumer and non-clinical neural interfaces available, such as the Neurosity and OpenBCI products, which aim to democratise the use of EEG sensors by offering low-cost hardware and simple-to-use software.

This thesis will focus on the software aspect of non-invasive neural interfaces and the challenges and complexities that arise in the given context of modern software engineering for production-ready end-user facing brain-computer interface applications.

\nomenclature[cnn]{CNN}{Convolutional neural networks}
\nomenclature[gan]{GAN}{Generative adversarial networks}
\nomenclature[bci]{BCI}{Brain-computer interface}
\nomenclature[ecog]{ECoG}{Electrocorticography}
\nomenclature[eeg]{EEG}{Electroencephalography}

% Kreative Darstellung der kreativen Idee und Zieldefinition. Wichtig ist hier, dass zwar eine klare Zielvorstellung (Umfang, Pflichtenheft mit Teilzielen, welche Qualität soll erreicht werden) vorhanden ist, aber der genaue Weg noch nicht exakt feststeht. Möglicher Umfang: 10 Prozent der Gesamtlänge.

\section{Project motivation}
\label{chapter1-project-motivation}

% Goal: Give an overview of the relevant scientific literature.
% What is known? Which questions remain open? Are there conflicts in the literature?
% - Several studies have shown that…
% - While some studies suggested that X (References) other studies pointed in the opposite
% direction (References)
% - The findings of some studies suggested that X (References). In contrast, other studies have shown that Y (References)
% - Two theories have been proposed to explain this phenomenon. According to theory X….
% - Three lines of research are relevant to this question. First,…
% - Devin et al. (2003) were one of the first who found evidence that…
% - Overall, it has remained unclear whether…
% - Taken together, it remains an open question whether…

\section{Research question}
\label{chapter1-research-question}

% Goal 1: State your research question
% - The goal of the present article is to…
% - The research question of the present article is…
% Goal 2: State your hypotheses/ predictions
% - We hypothesize that…
% - We predict that…
% - Two hypotheses are conceivable…
% - Our primary hypothesis is that…
% - Drawing on theory X, we hypothesize that…
% Goal 3: Give a rough outline of your research
% - We tested whether patients with a diagnosed major depression would report less depressive
% feelings after treatment X compared to a placebo treatment.
% - To test this hypothesis, we ….
% - To answer this question, we …
% - For this purpose, we conducted three studies. First,… Second,… Third,…
% - To shed more light on this, we used a combination of computer simulations and empirical
% studies. First, we used computer simulations to determine what behaviour would arise if theory
% 3
% X is true and what behaviour would arise if theory Y was true. Next, we tested these two
% predictions in three empirical studies.
% - The present article consists of three sections. In section 1, … In section 2,… In section 3

\section{Goals}
\label{chapter1-goals}

% o Does your introduction go from broad (topic) to specific (your research)?
% o Did you make clear why your topic is important?
% o Did you describe just enough research so that readers can understand how your
% research is the next logical step?
% o Did you make clear how your research is novel?
% o Did you make clear what is speculation, and what are established facts?
% o Did you add citations for everything you present as facts?
% o Did you use common scientific jargon?
% o Did you explain the jargon you use?