\chapter{Results}
% \graphicspath{{Chapter3/Figs/}{Chapter3/Figs/}}

\section{Steps Before The Analysis}
\label{steps-before-analysis}

- Before we analysed the data, we removed all reaction times that were larger than 2000 ms
(2\% of all observations) based on the assumption that such reaction times are unlikely to
reflect spontaneous responses.
- The data of two participants were excluded from the analyses because they did not complete
the whole study.
- Functional images were re-aligned, unwarped, corrected for slice timing, and spatially
smoothed using an 8 mm smoothing kernel.

\section{Main Results}
\label{main-results}

- First, we investigated whether X (research question)
- We used an Independent samples t test with groups as independent variable and the
depression score as dependent variable
- The results showed that the difference between the groups/ conditions was significant
- The results showed a significant correlation between…
- The results showed a significant interaction between…
- Specifically, the average depressions score was lower in the treatment group (M=3.45, SD = 2.18) compared to the placebo group (M=4.83, SD = 2.02).

\section{Figures And Tables}
\label{figures-and-tables}

Add figures to make important results easier to interpret or to provide more information. Use tables to add extensive amounts of information that would be hard to read in text-form.

\section{Goals}
\label{chapter3-goals}

o Did you describe everything that is needed to replicate your results?
o Did you describe all pre-processing steps before the main analyses?
o Did you mention to which research question each analysis belongs?
o Did you avoid interpreting your results?
o Did you add figures for making your key results easy to understand (or are they very simple)?
o Did you add tables for extensive amounts of (numerical) information?
