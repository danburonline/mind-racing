\chapter{Discussion}
% \graphicspath{{Chapter4/Figs/}{Chapter4/Figs/}}

\section{Summary}
\label{summary}

- Research question: Does the REFOCUS treatment work?
- Study: treatment group and placebo group with self-reported depression measured afterwards
- Findings: Depression was lower after the REFOCUS treatment compared to placebo

\section{Interpretation}
\label{interpretation}

- Explanation 1: REFOCUS treatment reduced depression
- Explanation 2: placebo treatment increased depression
- However, explanation 2 is unlikely because the same placebo was used in studies
A, B, C and there it didn’t increase depression

\section{Integration}
\label{integration}

- Previous research focused on the question of how unprocessed traumas could cause
depression
- We are the first who tested the “focus” explanation of depression

\section{Implications}
\label{implications}

- It is widely believed that depression is caused by unprocessed traumas
- Our findings offer a novel perspective: depression is caused by information
processing style
- Hence, new approach, new line of research to understand depression, new types of
treatment

\section{Limitations}
\label{limitations}

- We had no measure of depression prior to the treatment
- Reason: asking people to score their depression twice can lead to problems
(references)
- Consequence: we don’t know whether depression decreased in treatment
group (explanation 1) or increased in placebo group (explanation 2)
- However, as mentioned before, it is unlikely that depression increased
- Sample size was relatively low
- Reason: it’s hard to find enough people with a major depression
- However, our results were significant despite the low sample size. This speaks
to the effectiveness of the treatment

\section{Conclusion}
\label{conclusion}

- We investigated whether depression can be treated by training a positive focus
- Our findings confirm this
- Novel perspective on depression
- More research needed, more treatments that follow this approach should be developed

\section{Goals}
\label{chapter4-goals}

o Does your discussion go from specific (interpretation) to broad (implications)?
o Did you draw conclusions with reservations? (“A possible interpretation is…”)
o If you expressed a preference for one explanation over another, did provide clear
support for this preference?
o Did you describe how your research connects to previous research?
o Did you make clear what your research adds to existing research?
o Did you describe how your research advance our understanding or how they may inspire
future applications?
o Did you clearly admit limitations before qualifying them?
o Did you remind the reader of the value/implications of your research at the end?
o Did you include some pointers for future research? (optional)
