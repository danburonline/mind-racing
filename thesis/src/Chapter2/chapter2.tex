\chapter{Context}
% \graphicspath{{Chapter2/Figs/}{Chapter2/Figs/}}

\section{Landscape}
\label{chapter2-landscape}

% - Current landscape of non-invasive BCIs and their applications (take IDUN's competition overview as a starter point)
% - What it takes for a BCI to be mainstream-ready with IDUN as an example (talk with Simon and example of smart glasses)
% - Different ways of how a BCI works as of today
% - Current state of EEG-based games and applications for the normal user (steady state invoked etc.) (talk with Michel and do own research)
% - Passive brain-computer interfaces and user experience (read paper on passive brain-computer interfaces)
% - Current state of 3D on the web (WebGPU, pixel streaming, declarative graphics components, WebAssembly (and Rust), etc.)
% - Current state of VR and AR applications (Meta Quest, Snapchat, ARKit, etc.)
% - What production-grade software separates from others
% - Requirements of current BCI software offerings
% - Challenges and requirements of going with a cloud, API and SDK (privacy, security, IP, performance, scaling etc.)
% - General challenges of a N/CI

% Akademischer Hintergrund: Vorbilder, Referenzmaterial, Eingrenzung und vertiefte Begründung der Zielformulierung. Grundlagenforschung im Bereich vergleichbarer Medienprodukte. Kenntnis der fachspezifischen Theorien und Techniken. Hier muss umfassende Fach- und Handwerkskenntnis gezeigt werden. Es sollen möglichst viele Informationen verwendet werden, die helfen sollen Entscheidungen für die Erstellung des eigenen Medienprodukts zu treffen und Vorgehensweisen beim Entstehungsprozess des eigenen Medienprodukts zu begründen. Ebenso soll begründet werden, inwiefern die verwendeten Quellen für die Zielsetzung und deren Umsetzung geeignet sind.