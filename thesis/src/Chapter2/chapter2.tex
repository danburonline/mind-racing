\chapter{Research context}
\graphicspath{{Chapter2/Figs/}{Chapter2/Figs/}}

This chapter describes the context of the research question and the findings from the current literature. The reader is educated on the limitations of current non-invasive and unidirectional BCIs, the paradigm shift in developing cloud-based and production-ready software versus running software in a research environment, and the implications and hypotheses of web-based approaches to BCIs and 3D applications for VR and AR that relate to the future of software in general in the field of spatial computing.

\section{Limitations of BCIs}
\label{chapter2-limitations-of-bcis}

The capabilities of BCIs are not without limitations. In addition to the physical limitations, mainly in material science for the hardware aspects of BCIs, the author attempts to address a broader issue related to neuroscience that directly correlates to the software aspects.

\subsection{AGI-complete problem}
\label{chapter2-agi-complete-problem}

As outlined in Chapter 1, a holistic view of BCI must take into account the aspect of decoding measured neural data and making it intelligible to computer software. It is important to emphasise that the task of decoding neural data is different from decoding thoughts, which is a critical factor for software. Moreover, decoding neural data and extracting the thoughts behind it so that the software can understand them are disciplines on their own. For example, getting computers to recognise letters written on a photograph is a very different problem from reading the written words in the sentences (i.e. computer vision and natural language processing).

Another part is understanding the sentences and their meaning, as in natural language understanding (NLU). NLU is considered an AI-complete problem, which means that the difficulty of these computational problems is equivalent to solving the central problem of artificial general intelligence (AGI), assuming that general human-level intelligence is computational.

\subsection{Levels of abstraction}
\label{chapter2-levels-of-abstraction}

% TODO Add a figure
% TODO Add a citation of the spatial-visual thinking research

Imagine a red house in the middle of a forest. Depending on the individual thought process, one can imagine the house with a temporary vision in mind, as in visual-spatial thinking, or one can imagine it more verbally, such as conceptually comprehending each word after each other of what a red house is and that it is geographically located in a forest. It should also be addressed that different types of thoughts exist at different levels of abstraction and complexity. The movement of the right arm in the physical world is less abstract and easier to quantify than, for example, the visual image of a red house in a forest. The latter is more abstract and challenging to quantify than the former. It gets even more complicated when one imagines concepts that are inconceivable to visualise, such as the idea of a company. A company is only an abstract collective concept of humanity without a physical counterpart like the company itself and is, therefore, less straightforward and more complex to measure than the visual thought of the red house.

\subsection{Technical limitations}
\label{chapter2-technical-limitations}

% TODO Add example from the paper of the CPU

\subsection{Lack of data}
\label{chapter2-lack-of-data}

\subsection{Low risk and low impact}
\label{chapter2-low-risk-and-low-impact}

% technical limitations, collected from various neurons, relative high abstraction extracting

% difference between BCI interfacing and just collecting data

% we don't know where to search, we might have a large data set

% Different ways of how a BCI works as of today (talk with Michel), functional and anatomical differences in the brains, some people just work different

% brain states better for non-invasive (state of mind), detailed thoughts

% mobile fmri then might be able to collect detailed thoughts instead of state of mind

% no biggest data set is probably the thing that makes some stuff possible we never have thought about

% still lots of homework to do on the (computational, physiological) neuroscience side

% big inter and intra variability in the brains, brains can be influenced by everything: what u think about, what u ate, what happened, how you slept, what for memories etc.

% mostly hardware needs to be right for bidirectional

% not so reliable and robust, that's why low risk and low impact

% there are chemical processes involved: how much information is even involved in the electricity and not other part?

\section{BCI landscape}
\label{chapter2-research-landscape}

% Current landscape of non-invasive BCIs and their applications (take IDUN's competition overview as a starter point)

\subsection{Real-world BCI applications}
\label{chapter2-real-world-bci-applications}

% Current state of EEG-based games and applications for the normal user (steady state invoked etc.) (talk with Michel and do own research)

% everything is a neural interface, even our own body

% external stimuli, like a sound or light

% personal thoughts: active and passive; make difference of wearable and passive

% "sinnesorgan": motor stuff, visual, auditory, neurofeedback (learn how to control your thoughts, attention training for adhs e.g.) etc.

\subsection{Unobtrusive hardware and software}
\label{chapter2-unobtrusive-hardware-and-software}

% What it takes for a BCI to be mainstream-ready with IDUN as an example (talk with Simon and example of smart glasses)

\section{Active and passive BCIs for UX}
\label{chapter2-active-and-passive-bcis-for-ux}

% Passive brain-computer interfaces and user experience (read paper on passive brain-computer interfaces)

\section{Production-grade software}
\label{chapter2-production-grade-software}

% What production-grade software separates from others

\section{Cloud paradigm-shift}
\label{chapter2-cloud-paradigm-shift}

% Challenges and requirements of going with a cloud, API and SDK (privacy, security, IP, performance, scaling etc.)

\section{Web-first BCI architecture}
\label{chapter2-web-first-bci-architecture}

% Examples of BCI data going through web stacks, how to make it work (paper from Stegman)

\section{3D applications in the browser}
\label{chapter2-3d-applications-in-the-browser}

% Current state of 3D on the web (WebGPU, pixel streaming, declarative graphics components, WebAssembly (and Rust), etc.)

\section{Web-based AR and VR}
\label{chapter2-web-based-ar-and-vr}

% Current state of VR and AR applications (Meta Quest, Snapchat, ARKit, etc.)

\section{N/CI challenges}
\label{chapter2-nci-challenges}

% General challenges of a N/CI

% Akademischer Hintergrund: Vorbilder, Referenzmaterial, Eingrenzung und vertiefte Begründung der Zielformulierung. Grundlagenforschung im Bereich vergleichbarer Medienprodukte. Kenntnis der fachspezifischen Theorien und Techniken. Hier muss umfassende Fach- und Handwerkskenntnis gezeigt werden. Es sollen möglichst viele Informationen verwendet werden, die helfen sollen Entscheidungen für die Erstellung des eigenen Medienprodukts zu treffen und Vorgehensweisen beim Entstehungsprozess des eigenen Medienprodukts zu begründen. Ebenso soll begründet werden, inwiefern die verwendeten Quellen für die Zielsetzung und deren Umsetzung geeignet sind.

\nomenclature[nlu]{NLU}{Natural language understanding}
\nomenclature[agi]{AGI}{Artificial general intelligence}
